\documentclass[11pt]{bgcletter}
\usepackage{hyperref}

\name{Dr.\ Carlos A. Sierra}
\signature{ \vspace{-2cm} Carlos A. Sierra, PhD}
\email{csierra}
\telephone{6133}
\begin{document}
\begin{letter}{Dr. Takashi Asaeda\\
 Editor in Chief \\ Wetlands Ecology and Management}

\opening{Dear Dr. Aseda}
Thank you very much for your consideration of our manuscript and for the opportunity to submit a revised version. We made several changes to the manuscript based on the reviewers' comments. More prominently, we \dots 

 In the text below, we provide answers ({\color{blue} blue font}) to all reviewers' comments ({\it italics font}). 

{\bf Reviewer 4} \\

{\it I recognise the considerable efforts the authors have taken to undertake this study and complete a thorough laboratory analysis of the samples collected. Moreover, I also appreciate the necessity of understanding sediment dynamics and carbon accumulation with respect to mangrove systems, which this study addresses. Overall, what is lacking to me here in this manuscript is a sufficient level of detail in the general presentation and a recognition of the limitations of the study. This extends to a consistent failure to validly support various statements made throughout the manuscript by making appropriate reference to suitable literature sources. Furthermore, the detailed interpretations and contrasts between individual habitat `zones', which have been inferred from the basic statistical analyses used, are somewhat na\"ive.  The discussions section contains some very confidently-worded interpretations from the results, but does not provide sufficient information to support these
(be it either from literature sources or logical argument).  Finally, the authors appear to assume, throughout the introduction, methods and results section, that the reader will understand the rationale behind the field sampling and analytical methods used. Links between the hypotheses and why these methods were used are not given until well into the discussion section. The reader is left to try to figure this out for themselves.}

\begin{enumerate}
\item {\it Ln. 37 Describe what habitat or environmental characteristics make up a `basin', or `fringe' mangrove. Again - a map of the area would be very helpful here! (Figure 1 is not mentioned in the text).}

{\color{blue} Habitat and environmental characteristics of the basing and fringe mangroves are described in the Mathrials and Methods section. Here in the introduction, we are describing the main motivation of the study and therefore it is not the most appropriate place to describe the study site. Figure 1 is mentioned in the text in the Methods section.}

\item {\it Ln. 60-62 You need to provide information on why this is the case - and what your rationale for this method is. I presume - that 13C (CO2) is preferentially taken up by mangrove trees during photosynthesis, therefore accounting for higher ratios in the mangrove sediments? But you haven't given any sort of explanation for this. Your rationale does not seem to explain why allochthonous organic carbon sediments (derived from elsewhere in this catchment would not also be enriched with 13C). What is the rationale supporting this line of inquiry - and most importantly - where are the literature sources to validate it?}

\item {\it Ln 71-73 A map showing the location might be helpful. (Figure 1 is not mentioned in the manuscript text)}

{\color{blue} Figure 1 is a map that shows the location of the study site and the location of the plots. It was initially mentioned in section 2.2, but we added an earlier reference in section 2.1 to address this comment.}

\item {\it Ln. 77 How? Why?}

{\color{blue} The cause of the change of the Sin\'u river course is not well known, however it is well documented in Serrano (2004), which we cite in the manuscript for further information.}

\item {\it Ln 86 Using what apparatus? What diameter? What depth? How were the sample positions determined? How were your sample locations randomised? (This is a FUNDAMENTAL basic requirement for the parametric statistics you have used).}

{\color{blue} The diameter and depth were already given in the sentence above. We added a description of the soil corer and information on the random selection of points. We know this is fundamental, and quite obvious, so for this reason we didn't include it in the previous version, but we add it now to address this comment.}

\item {\it Ln 87-88 A map to show the layout of these plots? How big are they? How far are they spaced apart from each other? How were their original positions determined? What do the designations P, R and C actually refer to and why is it necessary to use them?}

{\color{blue} The location of the plots is presented in Figure 1. The size of the plots is 500 m$^2$. We added this information to the manuscript. Plot designations are relevant for consistency with previously published manuscripts form the same site. } 

\item {\it Ln 91-92 Why was this done? What was the rationale for choosing 20cm divisions? Where is a citation to support this?}

{\color{blue} This was done to see if we can detect differences by soil depth instead of analyzing the entire soil core. This is commonly done in soil science and usually does not require any specific justification.}


\item {\it Ln 93-96 I can't follow these two sentences. What is an 'end member'? Who or what were 'them'?}

{\color{blue} This sentence was reworded for clarity.}

\item {\it Ln 98 Where in the plot? How was this position determined? Was this at a position you simply liked the look of?}

{\color{blue} Actually outside the plot to avoid effects on the trees inside the plot. The location was three meters away from a randomly selected corner of the plot. }

\item {\it Ln 101-102 So was a (single) soil pit dug? (Ln. 98) Or were 3 soil pits dug? If these replicates came from a single soil pit - how can they be considered replicates?}

{\color{blue} No, two soil pits in plots P21 and C4. This is explicitly mentioned in the manuscript.}

\item {\it ln 105-107 Why were these protocols chosen? (Citations?)}

{\color{blue} These are standard soil processing protocols.}

\item {\it Ln 108 Manufacturer name and location.}

{\color{blue} Information add to text.}

\item {\it Ln 118 -120 Do you have a reference to support this inference and further detail to support why it is valid to assume that this method allows for indication of the 'origin of the carbon'?}

{\color{blue} This is basic knowledge on isotope source partitioning analysis. We assumed most reads are a familiar with this concept, however we added a reference to a textbook to satisfy reviewer curiosity. }

\item {\it Ln 131 What is this refinement method, why did you use it? Where is the justification in the literature for using it in this way?}

{\color{blue} This is simply the name of the technique. See \url{https://en.wikipedia.org/wiki/Rietveld_refinement}}

\item {\it Ln 131 ``Topas'' Is this a citation, a manufacturer name, a piece of software, or an alternative name for the piece of apparatus?}

{\color{blue} Topas is a software and Brucker the company who produces it. We added more information to the sentence for clarity.}

\item {\it Ln 132 Measured what of each? The clay? The mass of the sample? The Rietveld / Topas number?}

{\color{blue} Each sample.}

\item {\it Ln 136 - 139 I miss a rationale, here or in the introduction section, for why these analyses were necessary. What were you hoping to achieve by doing this? What were your hypotheses?}

{\color{blue} The rationale is presented in the introduction, where we mention that we use analyses of the structure and composition of the clay minerals to help in determining the origin of the stored carbon.}

\item {\it Ln 157 I remain more than a little confused by your analysis and the conclusions you have made from them. Your statistical analysis (an ANOVA, examining the variance in a range of response variables according to 4 categorical 'zones' within this habitat system: basin, fringe, river, sand) does not allow you to determine the statistical significance of differences between individual habitat zones, even if these contrasts appear (visually) in the figures. You will need to include a Tukey post-hoc HSD analysis of means to be able to validly make such inferences on a pair-wise basis.}

{\color{blue} Sorry for the confusion. In the ANOVA we didn't compare four categorical zones as the reviewer thought, but only the two groups of mangrove types: fringe and basin. Therefore the Tukey post-hoc test is not needed. }

\item {\it Ln 169-170 Error terms for these mean values??}

{\color{blue} Standard deviation values were added. }

\item {\it Ln 172-173 How can you make such a definitive inference, based on your observations alone? How do you know that X resulted in Y? (The reader is not convinced - as you haven't provided any explanation why!)}

{\color{blue} This is inferred by simply multiplying bulk density by \% organic carbon. Also, you can reach this conclusion by simply looking at the results of Table 1. We do not think we need to give a complex argumentation for this result since it is almost trivial. However, in the new version we point the reader to Table 1 to see this in more detail. }

\item {\it Ln 175 -177 I do not see how you can assume to have directly assessed such a contrast, based on the description of the statistical methods and analyses you have used.}

{\color{blue} Again, we are not comparing the four groups here, only the two mangrove types, and for this reason it is possible to express the result in the way we do in this paragraph.}

\item {\it Ln 188-190 The statistical test you have described in your methods section did not test for direct differences between your individual habitat zone categories, yet a statistically significant contrast between two individual categories is precisely what you are stating here. }

{\color{blue} i.d.e.m.}

\item Ln 229-230 But you haven't at any point described how or why it did this.

\item {\it Ln 236 Reference to support this claim? Surely you were sampling root-free soils? Or did your soil samples also contain live root biomass?}

{\color{blue} Reference added. }

\item {\it Ln 239 Wouldn't an increase in bulk density be expected with any soil? Why is this relevant to mangroves? Why is reporting this finding relevant?}

{\color{blue} Yes, this happens in most soils, but in mangroves bulk densities increase strongly due to hydrostatic pressure. However, we agree in that this sentence is not very relevant for the discussion on differences in bulk density among the two mangrove types.}

\item {\it Ln 251-254 On what basis? Do you have a references to support these rationales? It is quite a large assumption otherwise!}

{\color{blue} This is relatively well known for vertical carbon transport in soils. We added a few references to support this claim. }

\item Ln 281 - 288 Some, if not all of the information in this paragraph needs to be in the introduction / methods section. Without such validating information, until now, what you have done has remained pretty much meaningless to the reader.

\item {\it Ln 285 What kind of (plant) species - a sunflower? Wheat? Mangrove tree species?}

{\color{blue} We do not understand this comment by the reviewer. Line 285 explicitly says Rizophora mangle.}

\item {\it Ln 286 - 387 'and is *likely* to be' 
You haven't conducted sufficient analyses to be certain!
Here - and elsewhere - there are far too many 'definitive' statements like this in the manuscript, which are either unverifiable the your analyses you have conducted, or unsupported by relevant literature sources. Both these requirements should be the basic minimum for any peer-reviewed scientific study.}

{\color{blue} We beg to differ. In this paragraph we do not make any definitive statements as stated by the reviewer. When we use a word such as `likely', we are actually implying that this might be the case, but there is uncertainty. Notice that we use wording such as `this assumption is supported', `this could mean either ... or ...' , 'which would underpin', etc. These are all hedged statements that show our interpretation of the results and our favoring for a specific interpretation, but away from a definite true.}

\item {\it Ln 326 As opposed to being from a marine source? This (implied?) contrast should be made more clear.}

{\color{blue} The sentence was completed to make the contrast clear as suggested.}

\item {\it Ln 372 I am perplexed as to how you claim your analyses of bulk density were able to show this. Changes in BD were co-correlated with root density and organic carbon content, that's all. They cannot give any indication of the source of carbon in a soil.}

{\color{blue} Sure, bulk density has nothing to do with a determination of the origin of the carbon. Since this was the opening statement of the conclusions, we simply wanted to list all the analyses that were performed in the study. However, we removed bulk density from this paragraph to avoid confusion.}

\end{enumerate}

\newpage
{\bf Reviewer 5} \\

{\it The manuscript of Volkel et al. aims to investigate the origin of organic carbon in mangrove ecosystem in Colombia using stable isotopes and mineralogical analyses. The authors improve the manuscript following the suggestions of the reviewers. As the previous reviewers have already marked this article is of interest regionally but also as a contribution in a wider contest. I think that in this form the manuscript is almost ready for publication. I have just few comments.}

{\it Reviewer 2 highlighted as in the manuscript a statistical section on methodology was missing. The authors add a chapter on statistical methods without explain if before the ANOVA analysis they have checked the population for normality. This is an important point since a normal distribution is an assumption for ANOVA and many soil characters have not a normal distribution. In the file with your codes, I didn't find any code that is usually used to test for normality.}

{\it Why did you not add data on N/C ratio (or at least C/N ratio) as the reviewer 1 has suggested? You already have values on C and N, adding N/C ration should be a further indicator to support your results and conclusions.}

\vspace{2em}
We hope this new version adequately addresses reviewer's comments and it is now suitable for publication.

\closing{Sincerely, \\
 \includegraphics[scale=0.7]{../../../Documents/Personal/firma.jpg}
 }
 \end{letter}

 \end{document}
